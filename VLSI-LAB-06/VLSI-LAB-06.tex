\documentclass[a4paper, 12pt]{article}
\usepackage[UTF8]{ctex} 
\usepackage{geometry} 
\usepackage{enumitem} 
\usepackage{titlesec} 
\usepackage{xcolor} 
\usepackage{ulem} 
\usepackage{graphicx} 
\usepackage{float} 
\usepackage{amsmath} 
\usepackage{subcaption}
\usepackage{circuitikz}
\usepackage{hyperref} 
\usepackage{fancyhdr}

\geometry{left=2.5cm, right=2.5cm, top=2.5cm, bottom=2.5cm}

% 设置行距
\linespread{1.5}

% 设置段落间距
\setlength{\parskip}{0.5em}

\titleformat{\section}{\Large\bfseries}{\thesection}{1em}{}
\titleformat{\subsection}{\large\bfseries}{\thesubsection}{1em}{}

\setlist[enumerate,1]{label=\arabic*、, leftmargin=*}
\setlist[enumerate,2]{label=\arabic*., leftmargin=*}

% XeLaTeX下使用系统字体
\setCJKmainfont{SimSun}[AutoFakeBold=2.5] % 宋体
\setCJKsansfont{SimHei} % 黑体
\setCJKmonofont{FangSong} % 仿宋
\setCJKfamilyfont{kai}{KaiTi} % 楷体
\newcommand{\kai}{\CJKfamily{kai}}

% 配置hyperref
\hypersetup{
    colorlinks=true,
    linkcolor=black,
    urlcolor=blue,
    citecolor=blue
}

\begin{document}

% 标题
\begin{center}
    {\kai\LARGE\textbf{实验六\ \ booth编码乘法器}}
\end{center}

% 页眉页脚
\pagestyle{fancy}
\fancyhf{} 
\fancyhead[L]{\kai 实验六\ \ booth编码乘法器} % 左侧页眉
\fancyhead[R]{\href{https://github.com/yuchihatuntun/SYSU-MST-VLSI_Lab}{GitHub} | 23342107\ 徐睿琳は}                   % 右侧页眉
\fancyfoot[C]{\thepage}                       % 页脚居中页码
\renewcommand{\headrulewidth}{0.4pt}          % 页眉线宽度
\renewcommand{\footrulewidth}{0pt}            % 页脚线宽度

\begin{center}
    23342107\ 徐睿琳
\end{center}

\vspace{0.5cm}

% 实验要求
\section*{一、实验要求}

    设计booth编码乘法器,了解并掌握booth编码的基本原理。

% 实验目的
\section*{二、实验目的}
\begin{enumerate}
    \item 了解并掌握booth编码的基本原理。
    \item 学会设计booth编码的乘法器。
\end{enumerate}

% 实验内容
\section*{三、实验内容}
\begin{enumerate}
    \item 设计一个Radix-4 booth编码的4bit有符号乘法器,设计booth编码模块和利用编码信号得到部分积,最终通过加法器相加得到乘积,验证其功能正确。重点在于了解booth编码的基本原理和作用及其电路实现。
\end{enumerate}

% 实验步骤
\section*{四、实验步骤}

\subsection*{1. Booth编码乘法原理}

    Booth算法通过对乘数进行重新编码,减少部分积的数量,从而提高乘法运算的速度。对于Radix-4 Booth编码(基4 Booth算法),它将乘数每3位一组进行编码,每次处理2位,部分积数量减少一半。

    设乘数 $B$ 为 $N$ 位二进制补码数 $B = b_{N-1}b_{N-2}\dots b_1b_0$,并在最低位补 $b_{-1}=0$。Radix-4 Booth编码规则如下:

    \begin{equation}
        PP_i = \text{Operation} \times A \times 2^{2i}
    \end{equation}
    其中 $A$ 为被乘数,$PP_i$ 为第 $i$ 个部分积。编码依据 $b_{2i+1}, b_{2i}, b_{2i-1}$ 三位信号决定操作类型。

    \begin{table}[H]
        \centering
        \caption{Radix-4 Booth 编码表}
        \begin{tabular}{|c|c|c|c|c|}
            \hline
            $b_{2i+1}$ & $b_{2i}$ & $b_{2i-1}$ & 操作 (Operation) & 部分积生成 \\
            \hline
            0 & 0 & 0 & 0 & 0 \\
            0 & 0 & 1 & $+1$ & $+A$ \\
            0 & 1 & 0 & $+1$ & $+A$ \\
            0 & 1 & 1 & $+2$ & $+2A$ (左移1位) \\
            1 & 0 & 0 & $-2$ & $-2A$ (取反加1,左移1位) \\
            1 & 0 & 1 & $-1$ & $-A$ (取反加1) \\
            1 & 1 & 0 & $-1$ & $-A$ (取反加1) \\
            1 & 1 & 1 & 0 & 0 \\
            \hline
        \end{tabular}
    \end{table}

    对于4-bit有符号乘法器,乘数 $B$ 扩展为 $b_3b_2b_1b_0b_{-1}$,分为两组:
    \begin{itemize}
        \item 组0 ($i=0$): $b_1, b_0, b_{-1}$ $\rightarrow$ 生成 $PP_0$
        \item 组1 ($i=1$): $b_3, b_2, b_1$ $\rightarrow$ 生成 $PP_1$
    \end{itemize}
    最终乘积 $P = PP_0 + (PP_1 \ll 2)$。

\subsection*{2. Booth编码模块设计}

\begin{figure}[htbp]
  \centering
  \includegraphics[width=0.8\textwidth]{images/image-2025-12-24-22-17-46.png}
  \caption{Booth编码模块逻辑电路图}
  \label{fig:image-2025-12-24-22-17-46.png}
\end{figure}

    设计Booth编码模块,根据乘数的3位输入生成对应的部分积控制信号。使用组合逻辑电路实现。

\subsection*{3. Booth选择模块设计}

\begin{figure}[htbp]
  \centering
  \includegraphics[width=0.8\textwidth]{images/image-2025-12-24-22-18-11.png}
  \caption{Booth选择模块逻辑电路图}
  \label{fig:image-2025-12-24-22-18-11.png}
\end{figure}

    设计Booth选择模块,根据Booth编码信号选择相应的部分积。使用多路选择器实现。

\subsection*{4. 部分积加法电路设计}

\begin{figure}[htbp]
  \centering
  \includegraphics[width=0.8\textwidth]{images/image-2025-12-24-22-18-28.png}
  \caption{顶层乘法模块}
  \label{fig:image-2025-12-24-22-18-28.png}
\end{figure}

    设计部分积加法电路,将生成的部分积进行加法运算,得到最终乘积。使用全加器阵列实现。

% 实验结果
\section*{五、实验结果}

\begin{figure}[htbp]
  \centering
  \includegraphics[width=0.8\textwidth]{images/image-2025-12-24-22-18-51.png}
  \caption{Booth编码乘法器仿真波形图}
  \label{fig:image-2025-12-24-22-18-51.png}
\end{figure}

如图所示,仿真结果验证了设计的Radix-4 Booth编码乘法器的正确性。输入被乘数和乘数后,输出乘积符合预期结果。

% 总结
\section*{六、总结}

通过本次实验,我深入理解了Radix-4 Booth编码乘法器的工作原理及其在数字集成电路设计中的重要性。

    首先,在理论层面,我掌握了Booth算法的核心思想:通过对乘数进行重新编码(Recoding),将连续的“1”序列转换为“+1”和“-1”的操作,从而减少部分积(Partial Products)的数量。对于Radix-4 Booth算法,每两位乘数生成一个部分积,使得部分积的数量减少了一半(对于N位乘法器,部分积从N个减少到N/2个)。这显著降低了后续加法树(如Wallace树或压缩器阵列)的复杂度,从而提高了乘法器的运算速度并减少了面积和功耗。

    其次,在电路设计层面,我学会了如何将算法转化为具体的硬件模块:
    \begin{itemize}
        \item \textbf{编码模块(Booth Encoder):} 实现了根据相邻三位乘数生成控制信号(如单倍、双倍、取反等)的逻辑。
        \item \textbf{部分积生成模块(Partial Product Generator):} 利用编码信号和被乘数生成实际的部分积,特别是处理负数时的“取反加一”操作(补码表示)以及移位操作。
        \item \textbf{加法阵列:} 理解了符号位扩展(Sign Extension)在部分积相加时的重要性,确保了有符号数运算的正确性。
    \end{itemize}

    最后,通过仿真验证,我确认了设计的4-bit Booth乘法器能够正确处理正数、负数以及零的乘法运算。本次实验不仅锻炼了我的逻辑电路设计能力,也让我体会到了算法优化对硬件性能提升的关键作用,为后续设计更高位宽、更高性能的运算单元打下了坚实基础。

% 附录
\section*{附录}
\appendix
\renewcommand{\thesection}{附录\Alph{section}}

\section{相关代码}

\section{数据表格}

\section{其他材料}

\end{document}