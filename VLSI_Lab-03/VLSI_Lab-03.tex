\documentclass[a4paper, 12pt]{article}
\usepackage[UTF8]{ctex} 
\usepackage{geometry} 
\usepackage{enumitem} 
\usepackage{titlesec} 
\usepackage{xcolor} 
\usepackage{ulem} 
\usepackage{graphicx} 
\usepackage{float} 
\usepackage{amsmath} 
\usepackage{subcaption}
\usepackage{circuitikz}
\usepackage{hyperref} 
\usepackage{fancyhdr}

\geometry{left=2.5cm, right=2.5cm, top=2.5cm, bottom=2.5cm}

% 设置行距
\linespread{1.5}

% 设置段落间距
\setlength{\parskip}{0.5em}

\titleformat{\section}{\Large\bfseries}{\thesection}{1em}{}
\titleformat{\subsection}{\large\bfseries}{\thesubsection}{1em}{}

\setlist[enumerate,1]{label=\arabic*、, leftmargin=*}
\setlist[enumerate,2]{label=\arabic*., leftmargin=*}

% XeLaTeX下使用系统字体
\setCJKmainfont{SimSun}[AutoFakeBold=2.5] % 宋体
\setCJKsansfont{SimHei} % 黑体
\setCJKmonofont{FangSong} % 仿宋
\setCJKfamilyfont{kai}{KaiTi} % 楷体
\newcommand{\kai}{\CJKfamily{kai}}

% 配置hyperref
\hypersetup{
    colorlinks=true,
    linkcolor=black,
    urlcolor=blue,
    citecolor=blue
}

\begin{document}

% 标题
\begin{center}
    {\kai\LARGE\textbf{实验三\ \ 逻辑门的功耗优化及EDP优化}}
\end{center}

% 页眉页脚
\pagestyle{fancy}
\fancyhf{} 
\fancyhead[L]{\kai 实验三\ \ 逻辑门的功耗优化及EDP优化} % 左侧页眉
\fancyhead[R]{\href{https://github.com/yuchihatuntun/SYSU-MST-VLSI_Lab}{GitHub} | 23342107\ 徐睿琳}                   % 右侧页眉
\fancyfoot[C]{\thepage}                       % 页脚居中页码
\renewcommand{\headrulewidth}{0.4pt}          % 页眉线宽度
\renewcommand{\footrulewidth}{0pt}            % 页脚线宽度

\begin{center}
    23342107\ 徐睿琳
\end{center}

\vspace{0.5cm}

% 实验要求
\section*{一、实验要求}

了解动态功耗的影响因素,掌握CMOS逻辑门的动态功耗优化。

% 实验目的
\section*{二、实验目的}

\begin{enumerate}
    \item 了解NAND门的搭建方法;
    \item 了解延迟的定义及计算方法;
    \item 掌握如何得到最小延时
\end{enumerate}

% 实验内容
\section*{三、实验内容}
\begin{enumerate}
    \item 针对指定逻辑门,通过仿真波形或电路图分析,写出其逻辑表达式,并计算其在 100ns 内的能耗;
        
    \item 设计并实现与实验 A)中逻辑功能相同的电路,命名为 lab3\_mycell,要求其能耗-延时积(EDP)小于实验 A)中的逻辑门。输入信号的占空比分别为:A:0.7,B:0.5,C:0.2,D:0.1。

\end{enumerate}

% 实验步骤
\section*{四、实验步骤}

\subsection*{4.1\ \texttt{lab3\_a} 逻辑分析与能耗计算}

\begin{figure}[H]\centering
	\includegraphics[width=0.8\columnwidth]{assets/1.png}
	\caption{\texttt{lab3\_a} 测试电路图}
	\label{fig:lab3_a_test_circuit}
\end{figure}

首先,结合 \texttt{lab3\_a} 的测试电路图与仿真波形,分析输入信号 A、B、C、D 的连接方式及其对输出 Y 的影响。通过观察 100ns 内各输入组合对应的输出变化,归纳输出 Y 的逻辑表达式。最后,统计 100ns 内的能耗数据,为后续电路优化和 EDP 比较提供依据。

\subsection*{4.2\ \texttt{lab3\_mycell} 设计与功耗综合优化}

\subsubsection*{\texttt{lab3\_a} 电路设计分析}

\begin{figure}[H]\centering
	\includegraphics[width=0.6\columnwidth]{assets/4.png}
	\caption{\texttt{lab3\_a} 电路图}
	\label{fig:lab3_a_circuit}
\end{figure}

如图\ref{fig:lab3_a_circuit}可见,\texttt{lab3\_a} 采用了四输入与非门与非门级联的结构实现功能,但是肉眼可见其扇入较大,导致延时较大,能耗较高。因此,在设计 \texttt{lab3\_mycell} 时,应考虑减少扇入,提高电路的速度和能效。

\subsubsection*{\texttt{lab3\_mycell\_01} 电路设计}

\begin{figure}[H]\centering
	\includegraphics[width=0.6\columnwidth]{assets/5.png}
	\caption{\texttt{lab3\_mycell\_01} 电路图}
	\label{fig:lab3_mycell_01_circuit}
\end{figure}

\subsubsection*{\texttt{lab3\_mycell\_02} 电路设计}

\begin{figure}[H]\centering
	\includegraphics[width=1\columnwidth]{assets/6.png}
	\caption{\texttt{lab3\_mycell\_02} 电路图}
	\label{fig:lab3_mycell_02_circuit}
\end{figure}

\subsubsection*{\texttt{lab3\_mycell\_03} 电路设计(基于\texttt{lab3\_b}实现)}

\begin{figure}[H]\centering
	\includegraphics[width=1\columnwidth]{assets/7.png}
	\caption{\texttt{lab3\_mycell\_03} 电路图}
	\label{fig:lab3_mycell_03_circuit}
\end{figure}

% 实验结果
\section*{五、实验结果}

\subsection*{5.1\ \texttt{lab3\_a} 逻辑分析与能耗计算}

\subsubsection{逻辑分析}

\begin{figure}[H]\centering
	\includegraphics[width=0.9\columnwidth]{assets/2.png}
	\caption{\texttt{lab3\_a} 100ns 内仿真波形图}
	\label{fig:lab3_a_simulation_waveform}
\end{figure}

如图\ref{fig:lab3_a_simulation_waveform}所示,通过观察波形图可知,输出 Y 的逻辑表达式为:

\[
Y = A \cdot B \cdot C \cdot D
\]

即\texttt{lab3\_a}是一个四输入与门。

\subsubsection{能耗计算}

根据\texttt{(abs(integ(i("/V5/PLUS" ?result "tran") 0 1e-07)) * 1.1)}计算,在 100ns 内,\texttt{lab3\_a} 的能耗为:

\begin{figure}[H]\centering
	\includegraphics[width=0.8\columnwidth]{assets/3.png}
	\caption{\texttt{lab3\_a} 100ns 内能耗计算结果}
	\label{fig:lab3_a_energy_consumption}
\end{figure}

% 总结
\section*{六、总结}

% 附录
\section*{附录}
\appendix
\renewcommand{\thesection}{附录\Alph{section}}

\section{相关代码}

\section{数据表格}

\section{其他材料}

\end{document}