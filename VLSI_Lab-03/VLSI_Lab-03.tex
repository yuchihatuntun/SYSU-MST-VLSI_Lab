\documentclass[a4paper, 12pt]{article}
\usepackage[UTF8]{ctex} 
\usepackage{geometry} 
\usepackage{enumitem} 
\usepackage{titlesec} 
\usepackage{xcolor} 
\usepackage{ulem} 
\usepackage{graphicx} 
\usepackage{float} 
\usepackage{amsmath} 
\usepackage{subcaption}
\usepackage{circuitikz}
\usepackage{hyperref} 
\usepackage{fancyhdr}

\geometry{left=2.5cm, right=2.5cm, top=2.5cm, bottom=2.5cm}

% 设置行距
\linespread{1.5}

% 设置段落间距
\setlength{\parskip}{0.5em}

\titleformat{\section}{\Large\bfseries}{\thesection}{1em}{}
\titleformat{\subsection}{\large\bfseries}{\thesubsection}{1em}{}

\setlist[enumerate,1]{label=\arabic*、, leftmargin=*}
\setlist[enumerate,2]{label=\arabic*., leftmargin=*}

% XeLaTeX下使用系统字体
\setCJKmainfont{SimSun}[AutoFakeBold=2.5] % 宋体
\setCJKsansfont{SimHei} % 黑体
\setCJKmonofont{FangSong} % 仿宋
\setCJKfamilyfont{kai}{KaiTi} % 楷体
\newcommand{\kai}{\CJKfamily{kai}}

% 配置hyperref
\hypersetup{
    colorlinks=true,
    linkcolor=black,
    urlcolor=blue,
    citecolor=blue
}

\begin{document}

% 标题
\begin{center}
    {\kai\LARGE\textbf{实验三\ \ 逻辑门的功耗优化及EDP优化}}
\end{center}

% 页眉页脚
\pagestyle{fancy}
\fancyhf{} 
\fancyhead[L]{\kai 实验三\ \ 逻辑门的功耗优化及EDP优化} % 左侧页眉
\fancyhead[R]{\href{https://github.com/yuchihatuntun/SYSU-MST-VLSI_Lab}{GitHub} | 23342107\ 徐睿琳}                   % 右侧页眉
\fancyfoot[C]{\thepage}                       % 页脚居中页码
\renewcommand{\headrulewidth}{0.4pt}          % 页眉线宽度
\renewcommand{\footrulewidth}{0pt}            % 页脚线宽度

\begin{center}
    23342107\ 徐睿琳
\end{center}

\vspace{0.5cm}

% 实验要求
\section*{一、实验要求}

了解动态功耗的影响因素,掌握CMOS逻辑门的动态功耗优化。

% 实验目的
\section*{二、实验目的}

\begin{enumerate}
    \item 了解NAND门的搭建方法;
    \item 了解延迟的定义及计算方法;
    \item 掌握如何得到最小延时
\end{enumerate}

% 实验内容
\section*{三、实验内容}
\begin{enumerate}
    \item 针对指定逻辑门,通过仿真波形或电路图分析,写出其逻辑表达式,并计算其在 100ns 内的能耗;
        
    \item 设计并实现与实验 A)中逻辑功能相同的电路,命名为 lab3\_mycell,要求其能耗-延时积(EDP)小于实验 A)中的逻辑门。输入信号的占空比分别为:A:0.7,B:0.5,C:0.2,D:0.1。

\end{enumerate}

% 实验步骤
\section*{四、实验步骤}

\subsection*{4.1\ \texttt{lab3\_a} 逻辑分析与能耗计算}

\begin{figure}[H]\centering
	\includegraphics[width=0.8\columnwidth]{assets/1.png}
	\caption{\texttt{lab3\_a} 测试电路图}
	\label{fig:lab3_a_test_circuit}
\end{figure}

首先,结合 \texttt{lab3\_a} 的测试电路图与仿真波形,分析输入信号 A、B、C、D 的连接方式及其对输出 Y 的影响。通过观察 100ns 内各输入组合对应的输出变化,归纳输出 Y 的逻辑表达式。最后,统计 100ns 内的能耗数据,为后续电路优化和 EDP 比较提供依据。

\subsection*{4.2\ \texttt{lab3\_mycell} 设计与功耗综合优化}

在设计初期,我们对基准电路 lab3\_a(基于 4 输入或 8 输入 NAND 门直连反相器)的局限性进行分析:

\subsubsection*{\texttt{lab3\_a} 电路设计分析}

\begin{figure}[H]\centering
	\includegraphics[width=0.6\columnwidth]{assets/4.png}
	\caption{\texttt{lab3\_a} 电路图}
	\label{fig:lab3_a_circuit}
\end{figure}

大扇入 NAND 门存在严重的堆叠效应,为平衡 $N$ 管串联带来的导通电阻,必须将 NMOS 尺寸放大 $N$ 倍。

\textbf{EDP 瓶颈}:

\begin{itemize}
	\item \textbf{延时($D$)}:串联管导致巨大的寄生延时($P \propto N$),导致本征速度极慢。
	\item \textbf{能耗($E$)}:为了补偿速度而被迫增大的晶体管尺寸,带来了巨大的栅极电容和扩散电容,导致动态功耗激增。
\end{itemize}

为了找到能耗延时积最小的结构,我尝试了三种替代\texttt{lab3\_a}逻辑实现方式:

\subsubsection*{并行树形结构 (\texttt{lab3\_mycell\_01})}

\textbf{设计思路}:利用 De Morgan 定律,我将 AND4 分解为两级逻辑:\texttt{NAND2 // NAND2 -> NOR2},将关键路径上晶体管的\textbf{最大堆叠高度}从 4 管串联降低至 2 管串联。

\begin{figure}[H]\centering
	\includegraphics[width=0.6\columnwidth]{assets/5.png}
	\caption{\texttt{lab3\_mycell\_01} 电路图}
	\label{fig:lab3_mycell_01_circuit}
\end{figure}

降低能耗延时积 (EDP) 的首要手段是降低寄生参数,而非盲目增大驱动。

\begin{enumerate}
	\item \textbf{打断串联链}:利用 De Morgan 定律将 4 管串联(或 8 管串联)分解为多级逻辑。优势在于 mycell\_01 中最大的串联数仅为 2(NAND2 或 NOR2),相比于基准电路,其寄生延时($P$)减半。
	\item \textbf{降低无效电容 (Reducing $C$)}:由于串联数减少,NAND2 的 NMOS 只需放大 2 倍即可平衡电阻。意味着更小的栅电容和内部节点电容。直接降低了每一次翻转所需的电荷量($Q$),从而降低能耗($E$)。
\end{enumerate}

\paragraph{参数扫描设计}

由于工艺最小宽度为 120 nm,故参数 a 的理论下限为 120/330 ≈ 0.36(此处取 0.4),参数 o 的理论下限为 120/280 ≈ 0.43(此处取 0.5)。

对 a 与 o 进行粗略参数扫描,共设 36 个取样点:

\begin{figure}[H]\centering
	\begin{subfigure}{0.5\columnwidth}
		\centering
		\includegraphics[width=0.85\linewidth]{assets/8.png}
		\caption{lab3\_mycell\_nand2参数}
		\label{fig:nand2_params}
	\end{subfigure}
	\hfill
	\begin{subfigure}{0.48\columnwidth}
		\centering
		\includegraphics[width=\linewidth]{assets/9.png}
		\caption{lab3\_mycell\_nor2参数}
		\label{fig:nor2_params}
	\end{subfigure}
	\caption{参数扫描设计}
	\label{fig:parameter_sweep_design}
\end{figure}

\begin{figure}[H]\centering
	\includegraphics[width=1\columnwidth]{assets/10.png}
	\caption{36步参数扫描设置}
	\label{fig:parameter_sweep_setup}
\end{figure}

\subsubsection*{驱动能力权衡,引入链式结构(lab3\_mycell\_02)}

在仿真 mycell01 后,我发现树形结构虽然本征性能优异,但在驱动 64 倍大负载 时显得力不从心。最后一级 NOR2 需要做得非常大才能驱动负载。因此,我尝试了 \texttt{lab3\_mycell\_02}(NAND2-INV 组合的链式结构)。

\textbf{局限性分析}:虽然驱动力改善了,但级数过多增加了总的开关活动率,可能导致能耗回升。

\begin{figure}[H]\centering
	\includegraphics[width=1\columnwidth]{assets/6.png}
	\caption{\texttt{lab3\_mycell\_02} 电路图}
	\label{fig:lab3_mycell_02_circuit}
\end{figure}

\subsubsection*{\texttt{lab3\_mycell\_03} 电路设计(基于\texttt{lab3\_b}实现)}

\begin{figure}[H]\centering
	\includegraphics[width=1\columnwidth]{assets/7.png}
	\caption{\texttt{lab3\_mycell\_03} 电路图}
	\label{fig:lab3_mycell_03_circuit}
\end{figure}

% 实验结果
\section*{五、实验结果}

\subsection*{5.1\ \texttt{lab3\_a} 逻辑分析与能耗计算}

\subsubsection{逻辑分析}

\begin{figure}[H]\centering
	\includegraphics[width=0.9\columnwidth]{assets/2.png}
	\caption{\texttt{lab3\_a} 100ns 内仿真波形图}
	\label{fig:lab3_a_simulation_waveform}
\end{figure}

如图\ref{fig:lab3_a_simulation_waveform}所示,通过观察波形图可知,输出 Y 的逻辑表达式为:

\[
Y = A \cdot B \cdot C \cdot D
\]

即\texttt{lab3\_a}是一个四输入与门。

\subsubsection{能耗计算}

根据\texttt{(abs(integ(i("/V5/PLUS" ?result "tran") 0 1e-07)) * 1.1)}计算,在 100ns 内,\texttt{lab3\_a} 的能耗为:

\begin{figure}[H]\centering
	\includegraphics[width=0.8\columnwidth]{assets/3.png}
	\caption{\texttt{lab3\_a} 100ns 内能耗计算结果}
	\label{fig:lab3_a_energy_consumption}
\end{figure}

\subsection*{5.1\ \texttt{lab3\_mycell} 设计与功耗综合优化}

\subsubsection*{\texttt{lab3\_a} 参考能耗延时积}

\begin{figure}[H]
  \centering
  \includegraphics[width=0.8\textwidth]{images/image-2025-12-10-20-47-34.png}
  \caption{\texttt{lab3\_a} 能耗延时积结果}
  \label{fig:image-2025-12-10-20-47-34.png}
\end{figure}

\subsubsection*{\texttt{lab3\_mycell\_01}}

\begin{figure}[H]
  \centering
  \includegraphics[width=0.8\textwidth]{images/image-2025-12-10-20-49-25.png}
  \caption{粗扫参数设置}
  \label{fig:image-2025-12-10-20-49-25.png}
\end{figure}

因为宽度最小值是 120n,所以 a 最低不得低于120/330 = 0.4 , o 最低不得低于120/280 = 0.5

\begin{figure}[H]
  \centering
  \includegraphics[width=0.8\textwidth]{images/image-2025-12-10-20-48-58.png}
  \caption{能耗延时积扫描结果}
  \label{fig:image-2025-12-10-20-48-58.png}
\end{figure}

\paragraph{结果分析}

\begin{enumerate}
	\item 当参数 $a=0.4$ 时,能耗最低,但延时显著增加(相比于第二小的 $a=2.32$ 的曲线)。
	\item 对于所有参数曲线,能耗延时积的极值大致出现在 $o=6 \sim 8$ 的区间。
\end{enumerate}

因此,进一步开展了第二轮参数粗略扫描实验:

\paragraph{第二轮参数扫描设计}
针对参数 $a$ 与 $o$,本次扫描共设置 64 个取样点,具体参数配置如下:

\begin{figure}[H]
  \centering
  \includegraphics[width=0.8\textwidth]{images/image-2025-12-10-20-51-27.png}
  \caption{二次参数粗扫设置}
  \label{fig:image-2025-12-10-20-51-27.png}
\end{figure}

经参数扫描分析,如图\ref{fig:image-2025-12-10-20-53-16.png},当 $a=1$、$o=0.44$ 时,能耗延时积(EDP)达到 $1.3609z$,较基准电路 \texttt{lab3\_a} 略有降低,验证了优化结构在能耗与延时权衡上的有效性。

\begin{figure}
  \centering
  \includegraphics[width=0.8\textwidth]{images/image-2025-12-10-20-53-16.png}
  \caption{二次参数粗扫结果}
  \label{fig:image-2025-12-10-20-53-16.png}
\end{figure}

\subsubsection*{第三轮参数扫描与理论分析}

基于逻辑努力(Logical Effort)理论,计算电路在不同尺寸配置下的延时与能耗权衡,进一步优化能耗-延时积(EDP)。

\paragraph{最小延时尺寸计算}

根据逻辑努力理论,电路达到最小延时时的关键参数如下:

\begin{enumerate}
	\item \textbf{路径总努力} $H$:
	\[
		H = G \times F = \left(\frac{4}{3} \times \frac{5}{3}\right) \times 64 \approx 142.2
	\]
	其中 $G$ 为路径逻辑努力,$F$ 为路径电气努力。
	\item \textbf{最优级努力} $h$:
	\[
		h = \sqrt{H} \approx 11.9
	\]
	\item \textbf{中间级(NOR2)理论输入电容} $C_{\text{mid, speed}}$:
	\[
		C_{\text{mid, speed}} = \frac{g_{\text{NOR2}} \times C_{\text{load}}}{h} = \frac{1.67 \times 64}{11.9} \approx \mathbf{8.9}
	\]
	此时电路速度最快,但能耗较高。
\end{enumerate}

\paragraph{最小面积尺寸计算}

为实现最低能耗,需采用满足驱动要求的最小面积配置:

\[
	C_{\text{mid, min}} \approx \mathbf{1.0}
\]
即中间级(NOR2)输入电容与前级(NAND2)保持一致。

\paragraph{优化变量设定}

选取第二级 NOR2 的尺寸系数 $K$ 作为优化变量,具体设定如下:

\begin{itemize}
	\item \textbf{第一级(NAND2)}:输入电容固定为 $C_{\text{in}} = 1$(工艺约束)。
	\item \textbf{第二级(NOR2)}:尺寸系数 $K$ 扫描范围为 $[1.0,\, 10.0]$。
	\item \textbf{理论预测}:最小 EDP 通常出现在最小延时尺寸的 $60\% \sim 80\%$ 区间。
\end{itemize}

\paragraph{结果分析}

经参数扫描与理论分析,发现该方案的能耗-延时积(EDP)未能优于前述优化结构,整体性能表现不及上一轮设计。故本次结果不予详细展示。

\subsubsection*{\texttt{lab3\_mycell\_02}}

为实现驱动能力与能耗的权衡,采用逐级放大结构,并以级间放大倍数 $k$ 作为核心优化变量,使各级电路尺寸按比例递增。具体设计如下:

电路分为三级,每级均包含一个 NAND2 与一个反相器(INV),各级器件尺寸配置如下表所示:

\begin{table}[H]
	\centering
	\renewcommand{\arraystretch}{1.2}
	\begin{tabular}{cccccl}
		\hline
		\textbf{级数} & \textbf{器件} & \textbf{PMOS宽度 ($W_p$)} & \textbf{NMOS宽度 ($W_n$)} & \textbf{备注} \\
		\hline
		第1级 & NAND2 (I1) & $226\,\mathrm{nm}$ & $384\,\mathrm{nm}$ & 最小基准,固定 \\
			  & INV (I4)   & $330\,\mathrm{nm}$ & $280\,\mathrm{nm}$ & 单位基准,固定 \\
		第2级 & NAND2 (I5) & $k \times 226\,\mathrm{nm}$ & $k \times 384\,\mathrm{nm}$ & 一次放大 \\
			  & INV (I6)   & $k \times 330\,\mathrm{nm}$ & $k \times 280\,\mathrm{nm}$ & 一次放大 \\
		第3级 & NAND2 (I7) & $k^2 \times 226\,\mathrm{nm}$ & $k^2 \times 384\,\mathrm{nm}$ & 二次放大 \\
			  & INV (I8)   & $k^2 \times 330\,\mathrm{nm}$ & $k^2 \times 280\,\mathrm{nm}$ & 二次放大(驱动大负载) \\
		\hline
	\end{tabular}
	\caption{逐级放大结构各级器件尺寸配置}
	\label{tab:stage_amplification}
\end{table}

实验结果显示,该结构的能耗-延时积(EDP)最低约为 $1.44\,\mathrm{z}$,优化效果有限,未能显著优于前述树形结构方案。

\subsubsection*{\texttt{lab3\_mycell\_03}}

本设计同样采用逐级放大结构,以级间放大倍数 $k$ 作为核心优化变量,使各级器件尺寸按比例递增。具体器件配置如下:

\begin{table}[H]
	\centering
	\renewcommand{\arraystretch}{1.2}
	\begin{tabular}{cccccl}
		\hline
		\textbf{级数(路径位置)} & \textbf{器件编号} & \textbf{器件类型} & \textbf{PMOS宽度 ($W_p$)} & \textbf{NMOS宽度 ($W_n$)}  \\
		\hline
		第1级      & I0   & NAND2 & $226\,\mathrm{nm}$        & $384\,\mathrm{nm}$         \\
		辅助级     & I2   & INV   & $330\,\mathrm{nm}$        & $280\,\mathrm{nm}$          \\
		第2级      & I1   & NOR2  & $k \times 428\,\mathrm{nm}$ & $k \times 182\,\mathrm{nm}$        \\
		第3级      & I3   & NAND2 & $k^2 \times 226\,\mathrm{nm}$ & $k^2 \times 384\,\mathrm{nm}$  \\
		第4级      & I4   & INV   & $k^3 \times 330\,\mathrm{nm}$ & $k^3 \times 280\,\mathrm{nm}$  \\
		\hline
	\end{tabular}
	\caption{逐级放大结构各级器件尺寸配置}
	\label{tab:mycell03_stage_amplification}
\end{table}

最终最优化的能耗-延时积(EDP)约为 $891.392\,\mathrm{y}$,较前述设计方案有显著提升。

\begin{figure}[H]
  \centering
  \includegraphics[width=0.8\textwidth]{images/image-2025-12-10-20-59-41.png}
  \caption{逐级放大结构能耗-延时积(EDP)随放大倍数 $k$ 变化曲线}
  \label{fig:image-2025-12-10-20-59-41.png}
\end{figure}

% 总结
\section*{六、总结}

本次实验通过对四输入逻辑门电路的设计与优化,深入探究了CMOS电路中能耗(Energy)与延时(Delay)的权衡关系,并重点分析了能耗延时积(EDP)的优化策略。

\subsection*{6.1\ 实验数据对比}

通过对四种不同拓扑结构的电路进行仿真与参数扫描,得到的最佳性能指标对比如下表所示:

\begin{table}[H]
	\centering
	\renewcommand{\arraystretch}{1.3}
	\begin{tabular}{|c|c|c|c|c|}
		\hline
		\textbf{电路方案} & \textbf{拓扑结构} & \textbf{关键优化手段} & \textbf{最佳 EDP} & \textbf{相对优化率} \\
		\hline
		\texttt{lab3\_a} & NAND4 + INV & 基准电路 & $\approx 1.45\,\mathrm{z}$ & - \\
		\hline
		\texttt{mycell\_01} & Tree (NAND2-NOR2) & 降低堆叠高度 & $\approx 1.36\,\mathrm{z}$ & $6.2\%$ \\
		\hline
		\texttt{mycell\_02} & Chain (NAND2-INV) & 增强驱动能力 & $\approx 1.44\,\mathrm{z}$ & $0.7\%$ \\
		\hline
		\texttt{mycell\_03} & Hybrid (NAND-NOR-INV) & 逻辑努力分配 & $\approx 0.89\,\mathrm{z}$ & $\mathbf{38.6\%}$ \\
		\hline
	\end{tabular}
	\caption{各电路方案性能对比汇总}
	\label{tab:summary_comparison}
\end{table}

\subsection*{6.2\ 理论分析与经验总结}

\subsubsection{堆叠效应与逻辑努力}
实验结果验证了逻辑努力(Logical Effort)理论的核心观点。对于长串联堆叠的逻辑门(如 NAND4),其逻辑努力 $g$ 随输入数量显著增加:
\[
g_{\text{NAND}n} = \frac{n+2}{3}
\]
对于 \texttt{lab3\_a} 中的 NAND4,其 $g = (4+2)/3 = 2$,导致本征延时较大。而 \texttt{mycell\_01} 将其分解为 NAND2 ($g=4/3$) 和 NOR2 ($g=5/3$) 的组合,虽然级数增加,但单级延时降低,且有效减小了寄生电容。

\subsubsection{EDP 优化的本质}
能耗延时积 $EDP = E \times D$。
\begin{itemize}
	\item \textbf{延时优化}:主要依赖于路径电气努力 $H$ 的均匀分配。根据 $f = g \cdot h$,当各级级努力 $f$ 相等时,路径延时最小。
	\item \textbf{能耗优化}:主要受总开关电容 $\sum C_L$ 影响。过大的晶体管尺寸虽然降低延时,但会线性增加动态功耗 $P_{dyn} = \alpha f C V_{dd}^2$。
\end{itemize}

在 \texttt{mycell\_03} 中,通过引入级间放大系数 $k$,我们实际上是在寻找延时下降收益与能耗上升代价之间的平衡点:
\[
\frac{\partial (E \cdot D)}{\partial k} = 0
\]
实验发现,单纯追求最小延时(Logical Effort 推荐的 $h \approx 4$)往往导致能耗过高,最佳 EDP 点通常位于比最小延时尺寸更小的区域(即“欠驱动”状态)。

\subsection*{6.3\ 不足与改进方向}

\begin{enumerate}
	\item \textbf{漏功耗忽视}:本次实验主要关注动态功耗。随着工艺尺寸缩小(如本实验涉及的深亚微米工艺),漏电流功耗占比将显著提升。未来的优化模型应引入 $P_{static} = I_{leak} V_{dd}$。
	\item \textbf{输入向量依赖性}:实验中采用了固定的输入占空比。实际上,不同逻辑门的开关活动率 $\alpha$ 强依赖于输入信号的统计特性。更严谨的设计应针对最坏情况或平均情况进行加权优化。
	\item \textbf{版图寄生参数}:目前的仿真基于原理图,未考虑互连线电阻电容。在实际物理实现中,复杂的树形结构(如 \texttt{mycell\_01})可能因布线拥塞导致性能下降,需结合版图后仿真进行验证。
\end{enumerate}

\end{document}